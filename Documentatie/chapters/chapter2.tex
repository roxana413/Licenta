\chapter{Programarea dinamică}
Programarea dinamică este o tehnică de proiectare a algoritmilor utilizată pentru rezolvarea problemelor de optimizare.
Pentru a rezolva o anumită problemă folosind programarea dinamică, trebuie să identifiăm în mod convenabil mai multe subprobleme.
După ce alegem subproblemele, trebuie să stabilim cum se poate calcula soluția unei subprobleme în funcție de alte subprobleme.
Principala idee din spatele programării dinamice constă în stocarea rezultatelor subproblemele pentru a evita recalcularea lor de fiecare dată când sunt necesare. În general, programarea dinamică se aplică pentru probleme de optimizare pentru care algoritmii greedy nu produc în general soluția optimă.

\section{Probleme rezolvate cu ajutorul programării dinamice}
Urmatoarele probleme pot fi rezolvate cu ajutorul programarii dinamice:
\begin{itemize}
  \item \textbf {Problema șirului lui Fibonacci} 
  \item \textbf {Problema buturugii} 
  \item \textbf {Problema distanței de editare} 
  \item \textbf {Problema plătii unei sume de bani folosind numar minim de bancnote} 
  \item \textbf {Problema înmultirii optime a unui sir de matrici}
  \item \textbf {Problema de selecție a activităților cu profit maxim}
  
\end{itemize}
\subsection{Problema de selecție a activităților cu profit maxim}
Problema de selecție a activităților cu profit maxim este o problema de optimizare care returneaza pentru o listă de activități caracterizate prin timp de început, timp de sfârsit și profit,  ordonate dupa timpul de sfârșit o secventă de activități care nu se suprapun și al caror profit este maxim. 
\begin{verbatim}
    Input: Numarul de activități n = 4
    Detaliile activităților { timp de inceput, timp de incheiere,  profit}
       Activitate 1:  {1, 2, 50} 
       Activitate 2:  {3, 5, 20}
       Activitate 3:  {6, 19, 100}
       Activitate 4:  {2, 100, 200}
    Output: Profit-ul maxim este 250, pentru soluția optimă formata din
    activitatea 1 si activitatea 4.
\end{verbatim}
\subsection{Pseudocod}
\begin{verbatim}
struct Activitate {
    timp de început: int
    timp de încheiere : int
    profit : int 
}

function planificareActivitățiPonderate(activități):
    // activitățile sunt sortate crescător după timpul de încheiere
    
    // inițializăm un vector pentru a stoca profiturile maxime pentru fiecare activitate, fie dp un vector de dimensiune n, unde n este numărul de activități
    dp[0] = activități[0].profit
    solutie[0] = [activitati[0]] //un vector binar 0 - nu am ales activitatea si 1 - am ales activitatea
    solutiiOptime = solutie //stocam soluțiile optime la fiecare pas
    // Programare dinamică pentru a găsi profitul maxim
    pentru i de la 1 la n-1:
        // Găsește cea mai recentă activitate care nu intră în conflict cu activitatea curentă
        solutiaCuActivitateaI = [1] // selectam activitatea curenta
        ultimaActivitateNeconflictuală = găseșteUltimaActivitateNeconflictuală(activități, i)
        
        // Calculează profitul maxim incluzând activitatea curentă și excluzând-o
        profitulCuActivitateaCurenta = activități[i].profit
        dacă ultimaActivitateNeconflictuală != -1:
            profitulIncluderiiActivitățiiCurente += dp[ultimaActivitateNeconflictuală]
            solutiaCuActivitateaI = solutiiOptime[ultimaActivitateNeconflictuală] + solutiaCuActivitateaI
        dp[i] = maxim(profitulIncluderiiActivitățiiCurente, dp[i-1])
        
        daca profitulIncluderiiActivitățiiCurente > dp[i-1]:
            solutie = solutieCuActivitateaI
        else
            solutie = solutie + [0]
        solutiiOptime = solutiiOptime + solutie 
    // Returnează soluția si profitul maxim
    return solutie, dp[n-1]

funcție găseșteUltimaActivitateNeconflictuală(activități, indexCurent):
    pentru j de la indexCurent-1 la 0:
        dacă activități[j].timpDeÎncheiere <= activități[indexCurent].timpDeÎnceput:
            return j
    return -1



\end{verbatim}

\subsection{Cum funcționeaza programarea dinamică în cazul acestei probleme}
Pentru aceasta problemă se poate folosi programarea dinamică, deoarece aceasta poate fi împarțită în subprobleme, respectiv la fiecare pas putem forma o soluție parțiala optimă, de al carei profit ne putem folosi la urmatorul pas. 
La fiecare pas trebuie să selectăm o activitate în ordinea în care au fost declarate în secvența de intrare (formăm soluția parțială ce conține activitatea curentă), să cautam dacă există în fața acestora activități care nu se suprapun cu ele, daca da, concatenăm cu soluția partiala optimă
formată cu activități de pâna la activitatea cu care nu se suprapune, apoi dacă aceasta soluție partiala ce conține activitatea curentă are un profit strict mai mare decat cel optim anterior, o alegem, în caz contrar, alegam să nu selectam această activitate. 
La primul pas soluția parțiala optimă de lungime 1 este formata din Activitatea 1, iar profit-ul maxim este 50. 
La al 2-lea pas, deoarece activitatea 1 nu se suprapune cu activitatea 2 se obține soluția partiala optimă formata din Activitatea1 și Activitatea2, iar profitul optim la pasul 2 este 50 + 20 = 70, care este mai mare decat cel anterior (condiție necesară). 
La al 3-lea pas soluția parțială optimă este formata din Activitatea 1, Activitatea 2 și Activitatea 3, deoarece Activitatea 3 nu se suprapune cu activitatea 2 ( înseamnă că nu se suprapune cu soluția partiala de la al 2-lea pas, fiind ordonate dupa timpul se sfârsit) și facem o concatenare cu soluția parțiala de la pas-ul 2 , și totodata profitul pentru această soluție parțiala este strict mai mare decat cel de la pasul 2, noul profit optim devenind 170. 
La al 4-lea pas, la fel ca și la ceilalți formăm mai întâi o soluție partială care continea Activitatea 4. Selectăm Activitatea 4 și parcurgem secvența de activități (de la actvitatea 4 catre activitatea 1) data ca input pentru a putea găsi dacă există o activitate cu care aceasta să nu se suprapună. Singura activitate cu care nu se suprapune este activitatea 1, și soluția parțială ce conține Activitatea 4 este formata din Activitatea 4 si Activitatea 1. Apoi verificam daca profitul pentru aceasta soluție parțiala = 250 este strict mai mare decat profitul optim anterior = 170, adevărat. Acest lucru înseamnă că soluția parțiala optimă de lungime 4 care este și soluția problemei este formata din Activitatea 1 și Activitatea 4. 

\section{Avantajele programării dinamice față de celelalte tehnici de proiectare}
În ceea ce privește programarea dinamică și tehnica greedy, 
în ambele cazuri apare noțiunea de subproblemă și proprietatea de substructură optimă. De fapt, tehnica greedy poate fi gândită ca un caz particular de programare dinamică, unde rezolvarea unei probleme este determinată direct de alegerea greedy, nefiind nevoie de a
enumera toate alegerile posibile. Avantajele programării dinamice fața de tehnica greedy sunt:
\begin{itemize}
  \item \textbf {Optimizare Globală}: : Programarea dinamică are capacitatea de a găsi soluția optimă globală pentru o problemă, în timp ce algoritmii greedy pot fi limitați la luarea deciziilor locale care pot duce la o soluție suboptimală.
  \item \textbf {Flexibilitate}: Programarea dinamică poate fi utilizată pentru o gamă mai largă de probleme, inclusiv cele care implică restricții mai complexe sau soluții care necesită evaluarea mai multor posibilități. În comparație, algoritmii greedy sunt adesea limitați la problemele care pot fi rezolvate prin luarea deciziilor locale în fiecare pas.
  
\end{itemize}
Cu toate acestea, algoritmii greedy pot fi mai potriviți pentru problemele care permit luarea de decizii locale și producerea rapidă a unei soluții aproximative.